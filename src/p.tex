% !TeX root =./x2.tex
% !TeX program = pdfLaTeX
\chapter{Introduction to Introduction to Gr\"obner basis}
\label{chap:intro}
\section{An algebra defined by systems of generators and relations}
First we see some prototypical examples.
\subsection{Case 1: Complex numbers.}
What is a complex number (if you know real numbers)?

For example, let
\begin{align*}
  \alpha &= 3i^2 + 1,\\
  \beta &= -5i^5 + i + 1,\\
  \gamma &= i(4i^2 - i).
\end{align*}
These are complex numbers.
So, calculate them and check whether $\alpha=\beta$, $\beta=\gamma$,
and $\gamma=\alpha$.
For this question, we can answer as follows:
\begin{align*}
  \alpha&=3i^2+1=3(-1)+1=-2\\
  \beta&=-5i^5+i+1=-5(-1)^2i+i+1=-4i+1\\
  \gamma&=i(4i^2-i)=4i^3-i^2=4(-1)i-(-1)=-4i+1
\end{align*}
So $\alpha\neq \beta=\gamma$.


What do we do now?
\begin{enumerate}
\item Continue the following:
  \begin{enumerate}
  \item Expand as a polynomial in the indeterminat (i.e., variable) $i$.
  \item Substitute $i^2\denotes -1$.
  \end{enumerate}
  Every complex number become the form $x+yi$
  with some $x,y\in\RR$.
\item For $x,x',y,y'\in\RR$,
  \begin{align*}
      x+yi=x'+y'i \iff \begin{cases}x=x'\\y=y'.\end{cases}
  \end{align*}
  (In the other words,
  $\CC$ is a vetorspace over $\RR$ with a basis $\Set{1,i}$.
  So $\Set{1,i}$ is linearly indenepdent over $\RR$.)

  Hence, if we get this form, then we can check the equality by
  comparing the coefficients.
\end{enumerate}



\begin{remark}
  The form $x+yi$ is important.
  For example,
  \begin{align*}
  (x,y,z)=(x', y',z') \implies x+yi+zi^2=x'+y'i+z'i^2  
  \end{align*}
  is true. The converse is, however, false.
  The case where $(x,y,z)=(-1,0,0)$ and $(x',y',z')=(0,0,1)$
  is a counter example.
\end{remark}

\begin{remark}
  We call the following problem \emph{the indentification problem}:
  \begin{quotation}
    Check the equality of given two elements.
  \end{quotation}
\end{remark}


\subsection{Case 2: Square roots.}
What is $\sqrt{5}$ (if you know rational numbers)?

For example, let
\begin{align*}
  \alpha &= 3\sqrt{5}^2 + 1,\\
  \beta &= -\sqrt{5}^5 + 24\sqrt{5}+5,\\
  \gamma &= \sqrt{5}(\sqrt{5} - 1).
\end{align*}
Calculate them and check whether $\alpha=\beta$, $\beta=\gamma$,
and $\gamma=\alpha$.
For this question, we can answer as follows:
\begin{align*}
  \alpha &= 3\sqrt{5}^2 + 1=3\cdot 5+1=16,\\
  \beta &= -\sqrt{5}^5 + 24\sqrt{5}+5=-(5)^2\sqrt{5}+24\sqrt{5}+5=-\sqrt{5}+5,\\
  \gamma &= \sqrt{5}(\sqrt{5} - 1)=\sqrt{5}^2-\sqrt{5}=5-\sqrt{5}=-\sqrt{5}+5.
\end{align*}
So $\alpha\neq \beta=\gamma$.

What do we do now?
\begin{enumerate}
\item Continue the following:
  \begin{enumerate}
  \item Expand as a polynomial in the indeterminat (i.e., variable) $\sqrt{5}$.
  \item Substitute $\sqrt{5}^2\denotes 5$.
  \end{enumerate}
  Every complex number become the form $x+y\sqrt{5}$
  with some $x,y\in\QQ$.
\item For $x,x',y,y'\in\QQ$,
  \begin{align*}
      x+y\sqrt{5}=x'+y'\sqrt{5} \iff \begin{cases}x=x'\\y=y'.\end{cases}
  \end{align*}
  (In the other words,
  $\Set{1,\sqrt{5}}$ is linearly indenepdent over $\QQ$.)

  Hence, if we get this form, then we can check the equality by
  comparing the coefficients.
\end{enumerate}



\subsection{Case 3: Root of unity.}
What is  $\omega=-\frac{1}{2}+\frac{\sqrt{3}}{2}i$
(if you know real numbers)?

Note that
\begin{align*}
  \omega&=-\frac{1}{2}+\frac{\sqrt{3}}{2}i\\
  &=\cos(\frac{2\pi}{3})+\sin(\frac{2\pi}{3})i\\
  &=e^{\frac{2\pi}{3}i}.
\end{align*}
$x=\omega$ is a solution of the equation
\begin{align*}
  x^2+x+1=0.
\end{align*}
Hence $x=\omega$ is also a solution of the equation $x^3=1$.

For example, let
\begin{align*}
  \alpha &= \omega^2+4\omega+7,\\
  \beta &= -\omega^3 -2\omega,\\
  \gamma &= \omega(\omega-1).
\end{align*}
Calculate them and check whether $\alpha=\beta$, $\beta=\gamma$,
and $\gamma=\alpha$.
For this question, we can answer as follows:
\begin{align*}
  \alpha &= \omega^2+4\omega+7=(-\omega-1)+4\omega+7=3\omega+6,\\
  \beta &= -\omega^3 -2\omega= -\omega\omega^2 -2\omega=-\omega(-\omega-1)-2\omega=\omega^2-\omega=(-\omega-1)-\omega=-2\omega-1,\\
  \gamma &= \omega(\omega-1)=\omega^2-\omega=(-\omega-1)-\omega=-2\omega-1.
\end{align*}
So $\alpha\neq \beta=\gamma$.

What do we do now?
\begin{enumerate}
\item Continue the following:
  \begin{enumerate}
  \item Expand as a polynomial in the indeterminat (i.e., variable) $\omega$.
  \item Substitute $\omega^2\denotes -omega-1$.
  \end{enumerate}
  Every complex number become the form $x+y\omega$
  with some $x,y\in\RR$.
\item For $x,x',y,y'\in\RR$,
  \begin{align*}
      x+y\omega=x'+y'\omega \iff \begin{cases}x=x'\\y=y'.\end{cases}
  \end{align*}
  (In the other words,
  $\Set{1,\omega}$ is linearly indenepdent over $\RR$.)

  Hence, if we get this form, then we can check the equality by
  comparing the coefficients.
\end{enumerate}

\subsection{Case 4: Quvic roots.}
What is $\left(5^{\frac{1}{3}}\right)=\sqrt[3]{5}$
(if you know rational numbers)?

For example, let
\begin{align*}
  \alpha &= 3\left(5^{\frac{1}{3}}\right)^3+2\left(5^{\frac{1}{3}}\right)^2+4 \left(5^{\frac{1}{3}}\right)+ 7,\\
  \beta &= -\left(5^{\frac{1}{3}}\right)^6 -\left(5^{\frac{1}{3}}\right)^2+30,\\
  \gamma &= \left(5^{\frac{1}{3}}\right)^2(\left(5^{\frac{1}{3}}\right) - 1).
\end{align*}
Calculate them and check whether $\alpha=\beta$, $\beta=\gamma$,
and $\gamma=\alpha$.
For this question, we can answer as follows:
\begin{align*}
  \alpha &= \left(5^{\frac{1}{3}}\right)^3+2\left(5^{\frac{1}{3}}\right)^2+4 \left(5^{\frac{1}{3}}\right)+ 7\\
  &=5+2\left(5^{\frac{1}{3}}\right)^2+4 \left(5^{\frac{1}{3}}\right)+ 7
  =2\left(5^{\frac{1}{3}}\right)^2+4 \left(5^{\frac{1}{3}}\right)+ 12,\\
  \beta &= -\left(5^{\frac{1}{3}}\right)^6 -\left(5^{\frac{1}{3}}\right)^2+30\\
   &= -\left(5^{\frac{1}{3}}\right)^3\left(5^{\frac{1}{3}}\right)^3 -\left(5^{\frac{1}{3}}\right)^2+30\\
   &= -5\cdot 5 -\left(5^{\frac{1}{3}}\right)^2+30
   = -\left(5^{\frac{1}{3}}\right)^2+5,\\
  \gamma &= \left(5^{\frac{1}{3}}\right)^2(\left(5^{\frac{1}{3}}\right) - 1)\\
  &= \left(5^{\frac{1}{3}}\right)^3 - \left(5^{\frac{1}{3}}\right)^2
  =5 - \left(5^{\frac{1}{3}}\right)^2.
\end{align*}
So $\alpha\neq \beta=\gamma$.

What do we do now?
\begin{enumerate}
\item Continue the following:
  \begin{enumerate}
  \item Expand as a polynomial in the indeterminat (i.e., variable) $\left(5^{\frac{1}{3}}\right)$.
  \item Substitute $\left(5^{\frac{1}{3}}\right)^3\denotes 5$.
  \end{enumerate}
  Every complex number become the form $x+y\left(5^{\frac{1}{3}}\right)+z\left(5^{\frac{1}{3}}\right)^2$
  with some $x,y,z\in\QQ$.
\item For $x,x',y,y',z,z'\in\QQ$,
  \begin{align*}
      x+y\left(5^{\frac{1}{3}}\right)+z\left(5^{\frac{1}{3}}\right)^2=x'+y'\left(5^{\frac{1}{3}}\right) +z'\left(5^{\frac{1}{3}}\right)^2\iff \begin{cases}x=x'\\y=y'\\z=z'.\end{cases}
  \end{align*}
  (In the other words,
  $\Set{1,\left(5^{\frac{1}{3}}\right),\left(5^{\frac{1}{3}}\right)^2}$ is linearly indenepdent over $\QQ$.)

  Hence, if we get this form, then we can check the equality by
  comparing the coefficients.
\end{enumerate}


\section{Summary}
We consider four cases.
In each case,
we caluculate polynomial in an indeterminant with relations.
To solve identification problem,
we use the same method in these prototypical cases.
We will try to generalize these calculations.

\begin{quiz}
  Consider the golden number
  \begin{align*}
    \tau = \frac{1+\sqrt{5}}{2}.
  \end{align*}
  Then $\tau$ is a root of the irreducible polynomial
  \begin{align*}
    \tau^2-\tau-1.
  \end{align*}
  In the other words, $x=\tau$ is a solution of the equation
  \begin{align*}
    x^2-x-1=0.
  \end{align*}
  Let
  \begin{align*}
  \alpha &= \tau^3,\\
  \beta &= \tau^4-\tau^2-\tau,\\
  \gamma &= \tau(\tau^2-1).
  \end{align*}
Check whether $\alpha=\beta$, $\beta=\gamma$,
and $\gamma=\alpha$.
\end{quiz}

\chapter{Our problem}
\label{chap:ourproblem}
\section{Summary of \Cref{chap:intro}}
We consider the following:
\begin{enumerate}
\item
  A polynomial over $\RR$ in the determinat $i$
  with the relation $i^2+1=0$.
\item
  A polynomial over $\QQ$ in the determinat $\sqrt{5}$
  with the relation $\sqrt{5}^2-5=0$.
\item
  A polynomial over $\RR$ in the determinat $\omega$
  with the relation $\omega^2+\omega+1=0$.
\item
  A polynomial over $\QQ$ in the determinat $\left(5^{\frac{1}{3}}\right)$
  with the relation $\left(5^{\frac{1}{3}}\right)^3-5=0$.
\item
  A polynomial over $\RR$ in the determinat $\tau$
  with the relation $\tau^2-\tau-1=0$.
\end{enumerate}
In each case,
we solve the identification problem by the following strategy:
\begin{enumerate}
\item Continue the following:
  \begin{enumerate}
  \item Expand as a polynomial
  \item
    \label{basicstrategy:item:subs}
    Substitute the relation as the following form:
    \begin{align*}
      \underbrace{\text{monomial}}_{\text{(hightest degree)}}\denotes \underbrace{\text{polynomial}}_{\text{(lower degree)}}.
    \end{align*}
  \end{enumerate}
\item Obtain  a linear combination of linearly indepenedent elements.
\end{enumerate}

\begin{remark}
  The substitution in \Cref{basicstrategy:item:subs}
  is ``one-way''.
  By the substitution the dgree of polynomial decreases strictly.
  So, to calculate,
  we do not need any huristics.
\end{remark}

In \Cref{chap:ourproblem},
our motivation is the following question:
\begin{quotation}
  What about polynomials with the other relations?
\end{quotation}


\begin{remark}
  \label{rem:ourprob}
  In the prototypical cases,
  we consider the following relations:
\begin{enumerate}
\item
  $i^2+1=0$ (with $\RR$ coefficients).
\item
  $\sqrt{5}^2-5=0$ (with $\QQ$ coefficients).
\item
  $\omega^2+\omega+1=0$ (with $\RR$ coefficients).
\item
  $\left(5^{\frac{1}{3}}\right)^3-5=0$  (with $\QQ$ coefficients).
\item
  $\tau^2-\tau-1=0$ (with $\RR$ coefficients).
\end{enumerate}
We have some candidate of the meanings of ``the other'' in our motivation.
For example, we have the following:
\begin{enumerate}
\item
  \label{ourprob:item:1}
  In each prototipical cases,
  the relation is defined by an irreducible polynomial.
  What about non-irreducible polynomials?

\item
  \label{ourprob:item:2}
  In each prototypical cases,
  we consider the unique relation.
  What about multiple relations?

\item
  \label{ourprob:item:3}
  In each prototypical cases,
  we consider polynomials
  in one indeterminant.
  What about multivariable polynoimals?
\end{enumerate}
\end{remark}


\section{The case of one indeterminant}
Here we consider the case of  of one indeterminant, and
we see that our strategy works.

To see it, we define some notion.
\begin{definition}
  We call the following proceedure
  \defit{Division Algorithm}:
\begin{enumerate}
\item Continue the following:
  \begin{enumerate}
  \item Expand as a polynomial
  \item
    \label{basicstrategy:item:subs}
    Substitute the relation as the following form:
    \begin{align*}
      \underbrace{\text{monomial}}_{\text{(hightest degree)}}\denotes \underbrace{\text{polynomial}}_{\text{(lower degree)}}.
    \end{align*}
  \end{enumerate}
\item Obtain  a linear combination of linearly indepenedent elements.
\end{enumerate}
We call the monomial with the highest degree in the relation
the \defit{initial monomial} of the relation.
\end{definition}

\begin{remark}
  By each substitution,
  the degree of the polynoimal
  decrease strictly.
  Hence the degree of the polynomial
  will be less than the degree of the initial term(s) of the relation(s)
  So,
  for each starting polynomial
  (i.e., the  input of algorithm),
  this proceedure will stop
  in finitely many steps.

  Note that an ``algorithm'' means
  a proceedure which will stop in finitely many steps
  for each input.
\end{remark}

\subsection{In the case of a unique relation}

First we consider the case of a unique relation.
This corresponds to our question \Cref{ourprob:item:1} in \Cref{rem:ourprob}.

If our relation is
\begin{align}
  x^n\denotes \sum_{i=0}^{n-1}a_ix^i,
  \label{eq:rel:gen}
\end{align}
then $\Set{1,x,\ldots,x^{n-1}}$ is linearly independent.
In the algebra defined by the indeterminant $x$ with the relation \ref{eq:rel:gen},
If we use the division algorithm,
then we obtain a linear combination of $\Set{1,x,\ldots,x^{n-1}}$
from any polynomial by division algorithm.


\begin{remark}
  [For readers who know algebra]
  In each prototypical case,
  we consider a relation defined by an irreducible polynomial.
  So our algebra defined by the indeterminant with the relation
  is a field.
  Hence every nonzero element has its inverse,
  e.g., $i$ has $i^{-1}=-i$.
  If a  relation is not irreducible,
  then the algebra is not a field.
  The  indeterminant might not have its iverse.
  This is, however, no problem.
  Even in this case,
  the division algorithm works.
\end{remark}


\subsection{In the case of multiple relations}

Next we consider the case where we have some relations.
This corresponds to our question \Cref{ourprob:item:2} in \Cref{rem:ourprob}.

In this case,
at first, we modify our relations by division algorithm.
Then we obtain unique relation which
implies any other relations.
So we can apply the case of unique realtion to the relation.

For example, consider the relations
\begin{align*}
  \begin{cases}
    x^8-x^2=0\\
    x^6-x^2=0.
  \end{cases}
\end{align*}
By the following calculation
\begin{align*}
x^8-x^2  \xrightarrow{x^6\Denotes x^2} x^{4}-x^2
\end{align*}
of substitution,
we obtain the relation $x^{4}-x^2=0$
from $x^8-x^2=0$.
So our relations are
\begin{align*}
  \begin{cases}
    x^4-x^2=0\\
    x^6-x^2=0.
  \end{cases}
\end{align*}
Moreover,
the calculation
\begin{align*}
x^6-x^2=x^{4}x^{2}-x^2  \xrightarrow{x^4\Denotes x^2} x^2x^2-x^2=x^{4}-x^2  \xrightarrow{x^4\Denotes x^2} 0,
\end{align*}
we obtain the relation $0=0$
from
$x^6-x^2$.
So our relations become
\begin{align*}
  \begin{cases}
    x^4-x^2=0.
  \end{cases}
\end{align*}
Now we obtain unique relation.

\begin{remark}
  By substitution,
  the degree of the target relation become less thatn
  the degree of used relation.
  Hence, by this modification,
  all relations except one realtion become $0$.
  Hence we have a unique relation.
\end{remark}
\begin{remark}
  This modification of relations
  is equivalent to so-called sEuclidian Algorithm.
  Hence we can obtain a unique relation as
  the greatest common divisor of the relations.
\end{remark}
\begin{remark}
    [For readers who know algebra]
    A polynomial ring in one indeterminant over a field is a PID.
    Hence every ideal has a system of generators consisting of
    one element.
    Therefore we can always apply to the case of unique realtion.
\end{remark}



\section{The case of multiple indeterminants}
Here we consider the case of  of more than one indeterminant.
In this case, we have some problems
to apply our strategy works.
We see problems with examples.
\subsection{Initial monomial.}
In the case of one indeterminant,
the degree induces a total order over monomials.
Hence we can canonically select the initial term in each polynomial.
If we have more than one indeterminant,
then we have no canonical way to select.
\begin{example}
  \label{ex:degree}
  Which term in $x^5y+xy^5+x^4y^4$ is initial?

  If the degree means the degree of $x$,
  then $x^5y$ is initial.
  If the degree means the degree of $y$,
  then $xy^5$ is initial.
  If the degree means the sum of degrees,
  then $x^3y^4$ is initial.  
\end{example}


\begin{example}
  Which term in $x^2+xy+y^2$ is initial?

  If the degree means the sum of degrees,
  then all monomials are the same.
\end{example}

\begin{example}
  Which term in $x+xy$ is initial?

  If the degree means the degrees of $x$,
  then all monomials are the same.
\end{example}



\subsection{Order of substitutions}
\label{subsec:nonunique}
Assume that we can choose the initial monomial for each relation.
Even if so,
we have some problem if we have more than one relation.

\begin{remark}
    [For readers who know algebra]
    A polynomial ring in more than one indeterminant is not a PID.
    Hence some ideal has a system of generators consisting of
    more than one element.
\end{remark}



Consider the following two relations:
\begin{align}
x^4y^5-y&=0 \label{badex:rel:1}\\
x^6y^3-x^2y&=0 \label{badex:rel:2}   
\end{align}
Assume that
$x^4y^5$ is the initial monomial of \cref{badex:rel:1}\footnote{Note that the degree of $x^4y^5$ is greater than $y$ in any sense in \Cref{ex:degree}.},
and that
$x^6y^3$ is the initial monomial of \cref{badex:rel:2}\footnote{Note that the degree of $x^6y^3$ is greater than $x^2y$ in any sense in \Cref{ex:degree}.}.
Then
our relations for substitution are
\begin{align}
x^4y^5&\denotes y, \label{badex:relsub:1}\\
x^6y^3&\denotes x^2y. \label{badex:relsub:2}   
\end{align}

Let us calculate $x^7y^5$.

If we use \Cref{badex:relsub:1} at first,
then we obtatin $x^3y$
by the following substitution:
\begin{align*}
  x^7y^5=x^3\cdot x^4y^5
  \xrightarrow{x^4y^5\Denotes y}
  x^3y.
\end{align*}
We can not apply our relations to $x^3y$.
Hence we stop here.

If we use \Cref{badex:relsub:2} at first,
then we obtatin $x^3y^3$
by the following substitution:
\begin{align*}
  x^7y^5=xy^2\cdot x^6y^3
  \xrightarrow{x^6y^3\Denotes x^2y}
  xy^2\cdot x^2y=x^3y^3 .
\end{align*}
We can not apply our relations to $x^3y^3$.
Hence we stop here.

We obtain different final forms $x^3y$ and $x^3y^3$ by substitution.
Since  we obtain $x^3y$ and $x^3y^3$ from the same polynomial,
these should be the same in our algebra.
This means that these form are not useful for identification problem.

\section{Summary}
In the case of one indeterminat,
we can calculate, or solve identificationproblem,
by the division algorithm and Euclidean Algorithm.

We also want to calculate in the case of multi-indeterminants,
but na\"\i ve
strategy does not work.

\begin{quiz}
  Consider $\QQ[x]/\Braket{x^{10}+x^4-2,x^9-1,x^6-1}$,
  i.e.,
  the algebra defined by $x$ with the relations
  \begin{align*}
    x^{10}+x^4-2=0,\\x^9-1=0,\\x^6-1=0.
  \end{align*}
  Let
  \begin{align*}
    \alpha &= x^5 + x^6,\\
    \beta &= x^4 + x,\\
    \gamma &= 2x.
  \end{align*}
  Check whether $\alpha=\beta$, $\beta=\gamma$,
  and $\gamma=\alpha$.
\end{quiz}


\chapter{G\"obner basis}
\label{chap:defgb}
\section{Summary of \Cref{chap:ourproblem}}
We want to calculate in multiindeterminant case.
We have, however, problems:
\begin{enumerate}
\item
\label{gb:prob:1}
  We have no canonical way to select the initial monomial.
\item
\label{gb:prob:2}
  The results depends on choice of order of substitutions.
\end{enumerate}
In \cref{chap:defgb},
we give solution for them.


\section{Monomial order --- anser for \cref{gb:prob:1}}
In \cref{chap:defgb},
we consider polynomials in more than one indeterminant,
e.g., $x_1,\ldots,x_n$.
We use so-called multiindex notation defined as follows:
\begin{definition}
  For $\aaalpha=(\alpha_1,\ldots,\alpha_n)$
  and $\xxx=(x_1,\ldots,x_n)$,
  we define $\xxx^{\aaalpha}$ by
  \begin{align*}
    \xxx^{\aaalpha}=
    x_1^{\alpha_1}\cdots x_n^{\alpha_n}.
  \end{align*}
\end{definition}


In the case of single indeterminant,
degree induces a total order over (monic) monomials.
The order has nice property,
i.e., compatibility with the product operation.
Thanks to this property,
the substitution is ``one-way''.
We generalize not the degree but this total order.

\begin{definition}[Monomial order]
  We call $\leq$ a \defit{monomial order}
  if
  \begin{enumerate}
  \item $\leq$ is  a total oder on the set $\Set{\xxx^\aaalpha|\aaalpha\in\NN^n}$ of monic monomials.
  \item $\xxx^\aaalpha\leq \xxx^\bbbeta \implies \xxx^\gggamma\xxx^\aaalpha\leq \xxx^\gggamma\xxx^\bbbeta $.
  \item $\forall \xxx^\aaalpha$, $1\leq \xxx^\aaalpha$
  \end{enumerate}
\end{definition}

\begin{example}
  [Lexicographi order]
  We define $\xxx^\aaalpha < \xxx^\bbbeta $
  if there exists $i$ such that
  \begin{align*}
    &j<i\implies \alpha_j=\beta_j,\\
    &\alpha_i < \beta_i.
  \end{align*}
  Then $\leq$ is a monomial order.
\end{example}
\begin{proof}
  By definition,
  this is a total order on monomials.

  Let $\xxx^\aaalpha < \xxx^\bbbeta$ and
   $i$ satisfy
  \begin{align*}
    &j<i\implies \alpha_j=\beta_j,\\
    &\alpha_i < \beta_i.
  \end{align*}
  Then 
  \begin{align*}
    &j<i\implies \alpha_j+\gamma_j=\beta_j+\gamma_j,\\
    &\alpha_i+\gamma_j < \beta_i+\gamma_j.
  \end{align*}
  Hence $ \xxx^\gggamma\xxx^\aaalpha< \xxx^\gggamma\xxx^\bbbeta$.


  Let $\alpha\neq (0,\ldots,0)$ and $i=\min\Set{j|\alpha_j\neq 0}$.
  Then
   $i$ satisfies
  \begin{align*}
    &j<i\implies 0=\alpha_j,\\
    &0<\alpha_i.
  \end{align*}
  Hence $1=\xxx^{(0,\ldots,0)} < \xxx^\aaalpha$.
\end{proof}

\section{Gr\"obner basis --- anser for \cref{gb:prob:2}}

Fix a monomial order $\leq$.
For substitution,
we only use relations of the form
\begin{align*}
  \underbrace{\xxx^\aaalpha}_{\text{initial monomial}}
  \denotes
  \sum_{\bbbeta\colon\bbbeta < \aaalpha}
  a_\bbbeta \xxx^\bbbeta.
\end{align*}
If we use the symbol $\denotes$,
then we assume that the left hand side is the initial monomial of
the relation.
For $c\neq 0$,
we can transform a relation
\begin{align*}
  c\xxx^\aaalpha
  -
  \sum_{\bbbeta\colon\bbbeta < \aaalpha}
  a_\bbbeta \xxx^\bbbeta
  =0
\end{align*}
to our form
\begin{align*}
  \xxx^\aaalpha
  \denotes
  \frac{1}{c}\sum_{\bbbeta\colon\bbbeta < \aaalpha}
  a_\bbbeta \xxx^\bbbeta.
\end{align*}
uniquely.
Hence  we identify a polynomial
\begin{align*}
  c\underbrace{\xxx^\aaalpha}_{\text{initial monomial}}
  -
  \sum_{\bbbeta\colon\bbbeta < \aaalpha}
  a_\bbbeta \xxx^\bbbeta
\end{align*}
with the relation
\begin{align*}
  \underbrace{\xxx^\aaalpha}_{\text{initial monomial}}
  \denotes
  \frac{1}{c}\sum_{\bbbeta\colon\bbbeta < \aaalpha}
  a_\bbbeta \xxx^\bbbeta.
\end{align*}

\begin{definition}[S-polynomial]
  Consider two realtions
\begin{align*}
  \xxx^\aaalpha
  &\denotes
  \sum_{\bbbeta\colon\bbbeta < \aaalpha}
  a_\bbbeta \xxx^\bbbeta,\\
  \xxx^{\aaalpha'}
  &\denotes
  \sum_{\bbbeta'\colon\bbbeta' < \aaalpha'}
  a_{\bbbeta'} \xxx^{\bbbeta'}.
\end{align*}
Let
\begin{align*}
  r&=\xxx^{\aaalpha}-\sum_{\bbbeta\colon\bbbeta < \aaalpha} a_\bbbeta \xxx^\bbbeta,\\
  r'&=\xxx^{\aaalpha'}-\sum_{\bbbeta'\colon\bbbeta' < \aaalpha'} a_{\bbbeta'} \xxx^{\bbbeta'}.
\end{align*}
Let $\xxx^{\aaalpha+\gggamma}=\xxx^{\aaalpha'+\gggamma'}$
be  the least common multiplier for $\xxx^{\aaalpha}$ and $\xxx^{\aaalpha'}$.
In other words,
\begin{align*}
  \gamma_i&=\max\Set{0,\alpha_i'-\alpha_i},\\
  \gamma'_i&=\max\Set{0,\alpha_i'-\alpha_i}.
\end{align*}
We define the S-polynomial $S_{\leq}(r,r')$ of $r$ and  $r'$ by
\begin{align*}
  S_{\leq}(r,r')&=
  -\xxx^{\gggamma}\sum_{\bbbeta\colon\bbbeta < \aaalpha} a_\bbbeta \xxx^\bbbeta
  +\xxx^{\gggamma'}\sum_{\bbbeta'\colon\bbbeta' < \aaalpha'} a_{\bbbeta'} \xxx^{\bbbeta'}\\
&=\xxx^\gggamma r- \xxx^{\gggamma'}r'.
\end{align*}
\end{definition}

Let $R$ be a set of relations, i.e., polynomials.
For a polynomial $f$,
we write
\begin{align*}
  f\% R \leadsto f'
\end{align*}
to denote that
we obtain $f'$ from $f$
by conituuing substitution unless impossible.
\begin{remark}
If 
  \begin{align*}
  f\% R \leadsto f',
  \end{align*}
  then we can not apply division algorithm to $f'$.
  Hence
  \begin{align*}
    f\% R \leadsto f
  \end{align*}
  means we can not apply division algorithm to $f'$.
\end{remark}

\begin{remark}
  As see in \cref{subsec:nonunique},
  the following conditions
  dot not imply $f'=f''$:
  \begin{align*}
  &f\% R \leadsto f',\\
  &f\% R \leadsto f''.
  \end{align*}
\end{remark}

\begin{definition}
  Let $R=\Set{r_1,\ldots,r_l}$ be a set of relations.
  We call $R$ a \defit{Gr\"obner basis}
  if
  \begin{align*}
    S(r_i,r_j) \% R \leadsto 0
  \end{align*}
  for any $i,j$.
\end{definition}
\begin{theorem}
  Let $R=\Set{r_1,\ldots,r_l}$ be a Gr\"obner basis.
  Let $B$ be the set of monomials $\xxx^\aaalpha$
  such that any initial monomial $r_i$ does not divide $\xxx^\aaalpha$.
  \begin{enumerate}
  \item For each polynomial $f$,
    there uniquely exsists $f'$ such that
    \begin{align*}
      f\% R \leadsto f'.
    \end{align*}
  \item If   
    \begin{align*}
      f\% R \leadsto f',
    \end{align*}
    then $f'$ is a linear combination of $B$.
  \item $B$ is linearly independent.
 \end{enumerate}
\end{theorem}

\begin{remark}
  If $R$ is a Gr\"obner basis,
  then
  the algebra with relations $R$ is a vector space with the basis $B$.
  Each element in $B$ is called a standard monomial.
\end{remark}

This is a solution for \cref{gb:prob:2}.
We, however, have another problem:
How do we obtain Gr\"obner basis?


\begin{theorem}
  [Buchberger Algorithm]
  Let $G$ be a set of relations.
  We can obtain a Gr\"obner basis by continuing the following:
  If there exist $r$ and $r' \in G$
  such that $S_{\leq}(r,r')\%G\leadsto c\xxx^\aaalpha-\sum_{\bbbeta\colon \bbbeta<\alpha} a_\bbbeta \xxx^\bbbeta$ with $c\neq 0$,
  then append $\xxx^\aaalpha\denotes \frac{1}{c}\sum_{\bbbeta\colon \bbbeta<\alpha} a_\bbbeta \xxx^\bbbeta$
  to $G$.

  The algebra defined by indeterminants with original relations
  is the same as the algebra defined by indeterminants with resulting Gr\"obner basis.
\end{theorem}

\begin{remark}
  Buchberger Algorithm is Euclidian Algorithm  in the case of single indeterminant.
  Buchberger Algorithm is Gaussian Elimination
  in the case where the relations are polynoimals of degree one.
\end{remark}

\begin{example}
  Consider $\QQ[x,y]/\Braket{x^4y^5-y,x^6y^3-x^2y}$.
  Calculate Gr\"obner basis by Buchberger Algorithm
\end{example}

\begin{quiz}
  Consider $\QQ[x,y]/\Braket{x^4y^5-y,x^6y^3-x^2y}$.
  Let
  \begin{align*}
    \alpha &= x^5y^5,\\
    \beta &= x^7y,\\
    \gamma &= x^3y^3.
  \end{align*}
  Check whether $\alpha=\beta$, $\beta=\gamma$,
  and $\gamma=\alpha$.
\end{quiz}
